\begin{SCn}
	
\scnsectionheader{Авторский коллектив Стандарта OSTIS}
	
\scnheader{соавтор Стандарта OSTIS}
\scnidtf{\uline{член Авторского Коллектива Стандарта OSTIS}, направленного на развитие Технологии OSTIS и описание каждой текущей версии этой Технологии в виде соответствующего Стандарта}
\scnrelfromset{направления и принципы организации деятельности}{
\scnfileitem{Отслеживать и читать новые публикации по тематике, рассматриваемой в Стандарте OSTIS.
Близкими источниками для этого являются:
\begin{scnitemize}
	\item выпуски журналов:
	\begin{scnitemizeii}
		\item "Онтология проектирования"
	\end{scnitemizeii}
	\item материалы конференций:
	\begin{scnitemizeii}
		\item КИИ, Консорциум W3C
	\end{scnitemizeii}
	\item публикации, ключевыми терминами которых являются:
	\begin{scnitemizeii}
		\item формальная онтология
		\item онтология верхнего уровня
		\item семантическая сеть
		\item граф знаний
		\item графовая база данных
		\item смысловое представление знаний
		\item конвергенция в ИИ
	\end{scnitemizeii}
	\item стандарты
\end{scnitemize}
};
\scnfileitem{Фиксировать результаты знакомства с новыми публикациями по тематике, близкой Стандарту 	OSTIS, в Библиографии OSTIS, а также в основном тексте Стандарта OSTIS в виде соответствующих 		ссылок, цитат, сравнительного анализа};
\scnfileitem{Отслеживать текущее состояние всего текста Стандарта OSTIS, формировать предложения, направленные на развитие Стандарта OSTIS и на повышение темпов этого развития. Активно участвовать в обсуждении проблем развития Технологии OSTIS};
\scnfileitem{Максимально возможным образом увязывать персональную работу над Стандартом OSTIS с другими формами деятельности - научной, учебной, прикладной};
\scnfileitem{Указывать авторство своих предложений по дополнению и/или корректировке текущего текста Стандарта OSTIS};
\scnfileitem{Участвовать в рецензировании и согласовании предложений, представленных другими авторами Стандарта OSTIS}
}


\scnheader{Организация работ по развитию Стандарта OSTIS}
\scneqtoset{
\scnfileitem{Всем прочитать текст текущей версии Стандарта OSTIS и написать конструктивные замечания к доработке};
\scnfileitem{Каждому уточнить свой персональный план участия в работе над второй версией (с учетом своих интересов и диссертационной тематики)}
}




\scnheader{План работ по подготовке \textit{Стандарта OSTIS-2022}}
\scneqtoset{
\scnfileitem{Распределить разделы по авторам (для всех сотрудников)};
\scnfileitem{Составить четкий план доработки каждого раздела};
\scnfileitem{Организовать регулярные семинары по обсуждению развития каждого раздела}
}




\scnheader{Орг-План каждого члена авторского коллектива}
\scneqtoset{
\scnfileitem{Прочитать весь текст текущей версии Стандарта OSTIS};
\scnfileitem{Сформировать свое мнение о текущих недостатках и направлениях развития Стандарта (осознать свою ответственность как соавтора)};
\scnfileitem{Из оглавления последующей версии Стандарта определить те разделы, в развитии которых вы готовы активно участвовать};
\scnfileitem{По согласованию утвердить распределение авторов по разделам и определить \uline{иерархическую} структуру локальных рабочих коллективов}
\newline
\scnaddlevel{1}
\scnnote{По каждому рабочему коллективу определить перечень вопросов, требующих обсуждения и согласования}
\scnaddlevel{-1}
}
	
\end{SCn}