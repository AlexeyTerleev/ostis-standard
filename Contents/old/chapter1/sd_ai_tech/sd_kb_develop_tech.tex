\begin{SCn}

\scnsectionheader{Предметная область и онтология технологий разработки баз знаний}

\scnstartsubstruct

\scnheader{Предметная область технологий разработки баз знаний современных интеллектуальных компьютерных систем}
\scnsdmainclasssingle{технология разработки баз знаний}
\scnsdclass{***}
\scnsdrelation{***}

\scnheader{методология разработки баз знаний}
\scnexplanation{Методология разработки онтологий представляет собой набор инструкций и руководств, описывающих процесс выполнения сложных процедур разработки онтологий. Она детализирует различные задачи, как они должны быть выполнены, в каком порядке и каким образом осуществлять документирование работы по созданию онтологий.}
\scnsubdividing{методология, поддерживающая совместную коллективную разработку онтологии;методология, не поддерживающая совместную коллективную разработку онтологии}
\scnsubdividing{методология, зависимая от инструментария;методология, частично зависимая от инструментария;методология, независимая от инструментария}
\scnauthorcomment{Есть много классификаций методологий, можно добавить}
\scnhaselement{скелетная методология Ушолда и Кинга}
\scnhaselement{методология Грюнингера и Фокса (TOVE)}
\scnhaselement{METHONTOLOGY}
\scnhaselement{On-To-Knowledge (OTK)}
\scnhaselement{KACTUS}
\scnhaselement{DILIGENT}
\scnhaselement{SENSUS}
\scnhaselement{UPON}

\scnheader{средство разработки баз знаний}
\scnsuperset{среда разработки онтологий}
\scnaddlevel{1}
\scnsuperset{инструмент создания онтологий}
\scnaddlevel{1}
\scnhaselement{Protege}
\scnhaselement{NeON}
\scnhaselement{Co4}
\scnhaselement{Ontolingua}
\scnhaselement{OntoEdit}
\scnhaselement{OilEd}
\scnhaselement{WebOnto}
\scnaddlevel{-1}
\scnsuperset{инструмент отображения, выравнивания и объединения онтологий}
\scnaddlevel{1}
\scnhaselement{PROMPT}
\scnhaselement{Chimaera}
\scnhaselement{OntoMerge}
\scnhaselement{OntoMorph}
\scnhaselement{OBSERVER}
\scnhaselement{FCAMerge}
\scnhaselement{ONION}
\scnaddlevel{-1}
\scnsuperset{инструмент аннотирования на основе онтологий}
\scnaddlevel{1}
\scnhaselement{MnM}
\scnhaselement{SHOE}
\scnhaselement{Knowledge Annotator}
\scnaddlevel{-2}
\scnsuperset{библиотека многократно используемых компонентов баз знаний}
\scnaddlevel{1}
\scnhaselement{Protege ontology library}
\scnhaselement{Ontaria ontology directory}
\scnaddlevel{-1}
\scnsuperset{средство коллективной разработки баз знаний}

\scnheader{средство коллективной разработки баз знаний}
\scnhaselement{Collaborative Protege}
\scnhaselement{NeON}
\scnhaselement{Co4}
\scnexplanation{В онтологическом инжиниринге любая онтология рассматривается как результат согласованной деятельности группы специалистов о модели некоторой области знаний. Исходя из этого с развитием методов и средств в области инженерии знаний все большее внимание стало уделяться инструментальной поддержке процесса коллективной разработки баз знаний и онтологий.}
\scntext{назначение}{
\begin{scnitemize}
\item управление взаимодействием и коммуникацией между разработчиками;
\item контроль за доступом к текущим результатам совместного проектирования;
\item фиксация авторских прав на экспертные знания, переданные в общее пользование;
\item обнаружение ошибок проектирования и управление коррекцией ошибок;
\item конкурентное управление изменениями.
\end{scnitemize}
}
\scntext{проблемы}{
\begin{scnitemize}
\item отсутствие развитых средств автоматического редактирования и верификации баз знаний, в том числе оценки полноты и избыточности;
\item отсутствие единого механизма коллективного создания баз знаний, включающего в себя средства согласования вносимых изменений между разработчиками разного уровня ответственности, типологию ролей разработчиков;
\item недостаточный уровень расширяемости инструментов разработки.
\end{scnitemize}
}

\scnheader{технология разработки баз знаний}
\scnsuperset{Wiki-технология}
\scntext{проблемы}{Несмотря на достигнутые успехи в области создания баз знаний, остаются актуальными следующие проблемы:

\begin{scnitemize}
\item трудоемкость одновременного использования моделей представления различных видов знаний;
\item несовместимость уже разработанных компонентов баз знаний приводит к необходимости повторной разработки уже существующих решений;
\item изменения, вносимые в базу знаний, могут повлечь необходимость внесения существенных изменений в саму структуру базы знаний, особенно в случае динамических баз знаний;
\item несмотря на наличие достаточно развитых средств создания баз знаний, они не в полной мере обеспечивают комплексную поддержку (в том числе – информационную) коллектива разработчиков на всех стадиях проектирования базы знаний, а также не обладают достаточной гибкостью и расширяемостью;
\item существующие средства ориентированы, как правило, на какой-либо конкретный формат хранения знаний, что затрудняет перенос уже разработанной базы знаний на другую платформу интерпретации модели.
\end{scnitemize}

Основной причиной всех указанных проблем является отсутствие в рамках базы знаний интеллектуальной системы совместимости различных видов знаний, в том числе метазнаний. Совместимость различных видов знаний включает два аспекта: синтаксическую совместимость, что подразумевает унификацию формы представления знаний, и семантическую совместимость, что подразумевает однозначную и единую для всех фрагментов базы знаний трактовку используемых понятий. Кроме того, при модификации и расширении база знаний должна сохранять свою целостность и непротиворечивость.

Существующие подходы к разработке баз знаний, как правило, предполагают решение задачи обеспечения синтаксической совместимости знаний путем соединения разнородных моделей представления знаний, а также разработки новых интегрированных моделей и новых языков представления знаний. Разработка базы знаний таким способом приводит к дополнительным накладным расходам при интеграции и обработке разнородных знаний и, как следствие, к резкому увеличению трудозатрат при модификации таких баз знаний и добавлении новых видов знаний.

Попытки решения задачи обеспечения семантической совместимости раличных видов знаний в рамках разрабатываемых баз знаний связаны с построением онтологий верхнего уровня, однако, отсутствие единой формальной основы, обеспечивающей однозначную интерпретацию представляемых знаний и вводимых новых понятий, не привело к решению указанной задачи. Кроме того, существующие средства создания баз знаний предполагают, что процессы разработки и модификации базы знаний осуществляются отдельно от процесса ее использования, что приводит к дополнительному усложнению решения задачи обеспечения совместимости знаний различного вида.}

\scnendstruct

\end{SCn}