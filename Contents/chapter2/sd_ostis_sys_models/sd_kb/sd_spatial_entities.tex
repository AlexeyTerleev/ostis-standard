\begin{SCn}
	
	\scnsectionheader{\currentname}
	
	\scnstartsubstruct
	
	\scniselement{атомарный раздел}
	
	\scnheader{Предметная область пространственных сущностей и их форм}
	\scniselement{предметная область}
	
	\scnheader{пространственная сущность}
	\scnexplanation{\textbf{\textit{Пространственная сущность}} - любой класс \textit{материальных объектов}, временно находящихся в определенном положении в пространстве.}
	
	\scnheader{форма*}
	\scnrelfrom{первый домен}{пространственная сущность}
	
	\scnheader{система координат}
	\scniselement{пространственная сущность}
	\scnexplanation{\textbf{\textit{Система координат}} - \textit{величины}, определяющие положение \textit{точки} на \textit{плоскости} и в пространстве.}
	
	\scnheader{декартова система координат}
	\scnidtf {прямоугольная система координат}
	\scnsubset{система координат} 
	
	\scnheader{двумерная декартова система координат}
	\scnsubset{декартова система координат}

	\scnheader{трехмерная декартова система координат}	
	\scnsubset{декартова система координат}
	
	\scnheader{начало отсчёта*}
	\scnrelfrom{первый домен}{система координат}
	\scnrelfrom{второй домен}{точка}
	
	\scnheader{точка}
	\scniselement{пространственная сущность}
	\scnexplanation{\textbf{\textit{Точка}} - это неделимый элемент соответствующего математического пространства, определяемого в геометрии, математическом анализе и других разделах математики, не имеющий никаких измеримых характеристик, кроме координат.}
	
	\scnheader{прямая}
	\scniselement{пространственная сущность}	
	\scnrelfrom{включение}{точка}
	\scnexplanation{\textbf{\textit{Прямая}} - это линия, не имеющая неровностей, скруглений и углов, а также являющаяся бесконечной, не имеющей ни начала, ни конца.}
	
	\scnheader{отрезок}
	\scniselement{пространственная сущность}	
	\scnrelfrom{включение}{точка}
	\scnexplanation{\textbf{\textit{Отрезок}} - это множество, состоящее из двух различных \textit{точек} данной \textit{прямой} (которые называются концами \textbf{\textit{отрезка}}) и всех \textit{точек}, лежащих между ними.}
	
	\scnheader{плоскость}
	\scniselement{пространственная сущность}
	\scnexplanation{\textbf{\textit{Плоскость}} - это бесконечная поверхность, к которой принадлежат все \textit{прямые}, проходящие через какие-либо две \textit{точки} \textbf{\textit{плоскости}}.}
	
	\scnheader{длина}
	\scniselement{измеряемый параметр}
	
	\scnheader{расстояние*}
	\scniselement{квазибинарное отношение}
	\scnrelfrom{второй домен}{длина}
	
	\scnheader{толщина*}
	\scniselement{бинарное отношение}
	\scnrelfrom{первый домен}{пространственная сущность}
	\scnrelfrom{второй домен}{отрезок}
	
	\scnheader{высота*}
	\scniselement{бинарное отношение}
	\scnrelfrom{первый домен}{пространственная сущность}
	\scnrelfrom{второй домен}{отрезок}
	
	\scnheader{ширина*}
	\scniselement{бинарное отношение}
	\scnrelfrom{первый домен}{пространственная сущность}
	\scnrelfrom{второй домен}{отрезок}
	
	\scnheader{длина*}
	\scniselement{бинарное отношение}
	\scnrelfrom{первый домен}{пространственная сущность}
	\scnrelfrom{второй домен}{отрезок}
	
\end{SCn}