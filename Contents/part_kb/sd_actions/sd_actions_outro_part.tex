\begin{SCn}

\scnheader{следует отличать*}
\scnhaselementset{
    \scnmakevectorlocal{действие;класс действий};
    \scnmakevectorlocal{метод;класс методов};
    \scnmakevectorlocal{деятельность;вид деятельности}
}
\scnaddlevel{1}
\scnsubset{семейство подклассов*}
\scnnote{Все сущности, принадлежащие рассмотренным \textit{понятиям}, требуют достаточно детальной \textit{спецификации}. При этом не следует путать сами сущности и их \textit{спецификации}. Так, например, не следует путать \textit{действие} и \textit{задачу}, которая специфицирует (уточняет) это \textit{действие}. Особое место среди указанных понятий занимает понятие \textit{метода}, т.к. каждый конкретный \textit{метод}, с одной стороны, является \textit{спецификацией} соответствующего \textit{класса действий}, а, с другой стороны, сам нуждается в \textit{спецификации}, которая уточняет либо \textit{декларативную семантику} этого \textit{метода} (т.е. обобщенную декларативную формулировку класса задач, решаемых с помощью этого \textit{метода}), либо \textit{операционную семантику} этого \textit{метода}, (т.е. множество \textit{методов}, обеспечивающих \textit{интерпретацию} данного специфицируемого \textit{метода}) и тем самым "преобразует"{} специфицируемый \textit{метод} в \textit{навык}.}
\scnaddlevel{-1}


\scnheader{следует отличать*}
\scnhaselementvector{первый домен*(спецификация*)\\
\scnaddlevel{1}
\scnidtf{специфицируемая сущность}
\scnidtf{сущность, использование которой требует вполне определенной ее спецификации}
\scnsuperset{действие}
\scnsuperset{класс действий}
\scnsuperset{метод}
\scnsuperset{класс методов}
\scnsuperset{деятельность}
\scnsuperset{вид деятельности}
\scnaddlevel{-1};
второй домен*(спецификация*)\\
\scnaddlevel{1}
\scnidtf{спецификация}
\scnsuperset{задача}
\scnaddlevel{1}
\scnsuperset{декларативная формулировка задачи}
\scnaddlevel{1}
\scnidtf{семантическая формулировка задачи}
\scnaddlevel{-1}
\scnsuperset{процедурная формулировка задачи}
\scnaddlevel{1}
\scnidtf{функциональная формулировка задачи}
\scnaddlevel{-1}
\scnaddlevel{-1}
\scnsuperset{план действия}
\scnaddlevel{1}
\scnidtf{план}
\scnidtf{план выполнения действия}
\scnaddlevel{-1}
\scnsuperset{декларативная спецификация выполнения действий}
\scnaddlevel{1}
\scnidtf{иерархическая система подзадач}
\scnaddlevel{-1}
\scnsuperset{протокол}
\scnsuperset{результативная часть протокола}
\scnsuperset{обобщенная декларативная формулировка класса задач}
\scnsuperset{метод}
\scnsuperset{декларативная семантика метода}
\scnsuperset{операционная семантика метода}
\scnsuperset{модель решения задач}
\scnaddlevel{-1}
}
\scnaddlevel{1}
\scnnote{
    При этом следует отличать:
    \begin{scnitemize}
        \item спецификацию конкретного \textit{действия} (\textit{задачу}, \textit{план}, \textit{декларативную спецификацию выполнения действия}, \textit{протокол}, \textit{результативную часть протокола});
        \item спецификацию конкретной \textit{деятельности} (\textit{контекст}*, \textit{множество используемых методов}*);
        \item спецификацию \textit{класса действий} (\textit{обобщенную декларативную формулировку класса задач}, \textit{метод});
        \item спецификацию \textit{вида деятельности} (\textit{технологию});
        \item спецификацию \textit{метода} (\textit{декларативную семантику метода}, \textit{операционную семантику метода});
        \item спецификацию \textit{класса методов} (\textit{модель решения задач}).
    \end{scnitemize}
}
\scnaddlevel{-1}


\scnheader{следует отличать*}
\scnhaselementset{
    \scnmakevectorlocal{действие;класс действий, метод};
    \scnmakevectorlocal{деятельность;вид деятельности, технология}
}
\end{SCn}
