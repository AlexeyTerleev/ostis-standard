\begin{SCn}

\scnreltovector{конкатенация сегментов}{Уточнение понятия воздействия и понятия действия. Типология воздействий и действий;Уточнение понятия задачи. Типология задач;Уточнение семейства параметров и отношений, заданных на множестве воздействий, действий и задач;Предметная область и онтология субъектно-объектных спецификаций воздействий;Уточнение понятий плана сложного действия, класса задач, метода; Уточнение понятия навыка, понятия класса методов и понятия модели решения задач;Уточнение понятия деятельности, понятия вида деятельности и понятия технологии}

\scnheader{Предметная область воздействий, действий, методов, средств и технологий}
\scniselement{предметная область}

\scnhaselementlist{исследуемый класс первичных объектов исследования}{
воздействие\\
	\scnaddlevel{1}
	\scnidtf{\textit{процесс} воздействия одних \textit{сущностей} на другие}
	\scnsubset{процесс}
	\scnaddlevel{-1}
;действие\\
	\scnaddlevel{1}
	\scnidtf{\textit{процесс}, "осознанно"{} и целенаправленно выполняемый (управляемый) некоторой \textit{кибернетической системой}}
	\scnsubset{воздействие}
	\scnsubset{процесс}
	\scnaddlevel{-1}
\bigskip
;неосознанное воздействие
\bigskip
;действие, выполняемое в памяти субъекта действия
;действие, выполняемое во внешней среде субъекта действия
;рецепторное действие\\
	\scnaddlevel{1}
	\scnidtf{действие, выполняемое рецептором субъекта действия}
	\scnaddlevel{-1}
;эффекторное действие\\
	\scnaddlevel{1}
	\scnidtf{действие, выполняемое эффектором субъекта действия}
	\scnaddlevel{-1}
\bigskip
;элементарное действие \\
	\scnaddlevel{1}
	\scnidtf{действие, выполнение которого не требует его декомпозиции на взаимосвязанные поддействия}
	\scnaddlevel{-1}
;сложное действие
;легко выполнимое сложное действие\\
	\scnaddlevel{1}
	\scnidtf{сложное действие, которое известно, как выполнять}
	\scnaddlevel{-1}
;интеллектуальное действие\\
	\scnaddlevel{1}
	\scnidtf{сложное действие, для которого априори не известно, как его выполнять}
	\scnaddlevel{-1}
\bigskip
;индивидуальное действие\\
	\scnaddlevel{1}
	\scnidtf{действие, выполняемое индивидуальной кибернетической системой}
	\scnaddlevel{-1}
;коллективное действие\\
	\scnaddlevel{1}
	\scnidtf{действие, выполняемое коллективом кибернетических систем (многоагентной системой)}
	\scnaddlevel{-1}
\bigskip
;планируемое действие
;инициированное действие
;выполняемое действие
	\scnaddlevel{1}
	\scnidtf{активное действие}
	\scnaddlevel{-1}
;прерванное действие
	\scnaddlevel{1}
	\scnidtf{выполняемое действие, находящееся в состоянии прерывания}
	\scnaddlevel{-1}
;выполненное действие
;отмененное действие
\bigskip
;действие с очень высоким приоритетом
;действие с высоким приоритетом
;действие со средним приоритетом
;действие с низким приоритетом
;действие с очень низким приоритетом}
\bigskip

\scnhaselementlist{исследуемый класс классов первичных объектов исследования}{
осмысленность воздействия\scnsupergroupsign
;длительность воздействия\scnsupergroupsign
;место выполнения действия\scnsupergroupsign
;функциональная сложность действия\scnsupergroupsign
;многоагентность действия\scnsupergroupsign\\
	\scnaddlevel{1}
	\scnidtf{коллективность субъекта действия}
	\scnaddlevel{-1}
;текущее состояние действия\scnsupergroupsign
;приоритет действия\scnsupergroupsign\\
	\scnaddlevel{1}
	\scnidtf{важность действия\scnsupergroupsign}
	\scnaddlevel{-1}
;срочность действия\scnsupergroupsign
\bigskip
;класс действий
;класс функционально эквивалентных действий\scnsupergroupsign
;класс логически эквивалентных действий\scnsupergroupsign
;класс семантических эквивалентных задач\scnsupergroupsign
;класс логически эквивалентных задач\scnsupergroupsign
;класс задач, для которого существует общий метод их решения\scnsupergroupsign
;класс аналогичных семантически элементарных процессов воздействия\scnsupergroupsign\\
	\scnaddlevel{1}
	\scnidtf{класс однотипных семантически элементарных воздействий\scnsupergroupsign}
	\scnaddlevel{-1}}

	
\scnhaselementlist{исследуемый класс классов}
{отношение, заданное на множестве* (действие)\\
	\scnaddlevel{1}
	\scnidtf{отношение, заданное на множестве действий}
	\scnaddlevel{-1}
;отношение, заданное на множестве* (задача)
;параметр, заданный на множестве* (действие)
;параметр, заданный на множестве* (задача)}
\scnaddlevel{1}
\scnsourcecommentpar{Здесь указаны классы классов, которые не являются классами классов \uline{первичных} объектов исследования}
\scnaddlevel{-1}
\bigskip

\scnhaselementlist{исследуемое отношение, заданное на множестве первичных объектов исследования}{
субъект\scnrolesign\\
    \scnaddlevel{1}
    \scnidtf{воздействующая сущность\scnrolesign}
    \scnaddlevel{-1}
;объект\scnrolesign\\
    \scnaddlevel{1}
    \scnidtf{воздействуемая сущность\scnrolesign}
    \scnaddlevel{-1}
;посредник\scnrolesign
;медиатор\scnrolesign
;спецификация воздействия*\\
	\scnaddlevel{1}
	\scnsuperset{спецификация действия*}
	\scnaddlevel{-1}
;спецификация действия*\\
	\scnaddlevel{1}
	\scnsuperset{задача*}
		\scnaddlevel{1}
		\scnsubdividing{декларативная формулировка задачи*;процедурная формулировка задачи*}
		\scnaddlevel{-1}
		\scnsuperset{план сложного действия*}
		\scnsuperset{декларативная спецификация выполнения сложного действия*}
		\scnsuperset{протокол*}
		\scnsuperset{результативная часть протокола*}
	\scnaddlevel{-1}
;декларативная формулировка задачи*
;процедурная формулировка задачи*
;план сложного действия*
;декларативная спецификация выполнения сложного действия*
;протокол*
;результативная часть протокола*}
\bigskip

\scnhaselementlist{исследуемое отношение}{спецификация класса действий*
;спецификация метода*
;спецификация класса методов*
;спецификация деятельности*
;спецификация вида деятельности*
}
\scnaddlevel{1}
	\scnsourcecommentpar{Здесь указаны исследуемые отношения, которые заданы не на множестве первичных объектов исследования}
\scnaddlevel{-1}
\bigskip

\scnhaselementlist{исследуемый класс структур, специфицирующих первичные объекты исследования}{
задача\\
	\scnaddlevel{1}
	\scnidtf{структура (sc-конструкция), содержащая достаточную информацию для выполнения соответствующего (специфицируемого) действия}
	\scnaddlevel{-1}
;декларативная формулировка задачи\\
	\scnaddlevel{1}
	\scnidtf{семантическая спецификация действия}
	\scnaddlevel{-1}
;процедурная формулировка задачи\\
	\scnaddlevel{1}
	\scnidtf{функциональная спецификация действия}
	\scnaddlevel{-1}
;план сложного действия\\
	\scnaddlevel{1}
	\scnidtf{план выполнения сложного действия}
	\scnaddlevel{-1}
;процедурный план сложного действия
;непроцедурный план сложного действия\\
	\scnaddlevel{1}
	\scnidtf{декларативный план сложного действия}
	\scnidtf{иерархическая система подзадач заданной сложной задачи}
	\scnaddlevel{-1}}
\bigskip

\scnhaselementlist{исследуемый класс структур}{метод\\
	\scnaddlevel{1}
	\scnidtf{спецификация класса сложных действий}
	\scnaddlevel{-1}
;денотационная семантика метода
;операционная семантика метода
;навык
;модель решения задач
;технология}
\bigskip

\scnhaselementlist{вводимое, но не исследуемое понятие}
{действие, выполняемое в памяти ostis-системы
;действие, выполняемое ostis-системой в своей внешней среде
;рецептурное действие ostis-системы
;эффекторное действие ostis-системы
;sc-агент\\
	\scnaddlevel{1}
	\scnidtf{внутренний субъект ostis-системы}
	\scnidtf{субъект, реализующий действия, выполняемые в памяти ostis-системы}
	\scnaddlevel{-1}}
\bigskip

\scnhaselementlist{используемое понятие, исследуемое в другой предметной области и онтологии}{
кибернетическая система\\
	\scnaddlevel{1}
	\scnidtf{сущность, обладающая способностью быть субъектом различного вида действий}
	\scnaddlevel{-1}
;компьютерная система\\
	\scnaddlevel{1}
	\scnidtf{искусственная кибернетическая система}
	\scnsubset{кибернетическая система}
	\scnaddlevel{-1}
;интеллектуальная компьютерная система\\
	\scnaddlevel{1}
	\scnsubset{компьютерная система}
	\scnsuperset{ostis-система}
	\scnaddlevel{-1}
;человек\\
	\scnaddlevel{1}
	\scnsubset{кибернетическая система}
	\scnaddlevel{-1}
;ostis-система
;спецификация*
	\scnaddlevel{1}
	\scnidtf{быть спецификацией (описанием, семантической окрестностью заданной сущности*)}
	\scnidtf{семантическая окрестность*}
	\scnaddlevel{-1}}

\bigskip
\scnrelfromlist{библиографический источник}{\scncite{Martynov1984}\\
	\scnaddlevel{1}
		\scnciteannotation{Martynov1984}
	\scnaddlevel{-1};
	\scncite{Ikeda1998}\\
	\scnaddlevel{1}
		\scnciteannotation{Ikeda1998}
		\scnrelfrom{ключевой знак}{онтология классов задач}
		\scnaddlevel{1}	
			\scnidtf{задачная онтология}
			\scnidtf{онтология классов задач, решаемых в данной предметной области}
		\scnaddlevel{-1}
	\scnaddlevel{-1};
	\scncite{Studer1996}\\
	\scnaddlevel{1}
		\scnciteannotation{Studer1996}
	\scnaddlevel{-1};
	\scncite{Benjamins1999}\\
	\scnaddlevel{1}
		\scnciteannotation{Benjamins1999}
	\scnaddlevel{-1};
	\scncite{Chandrasekaran1999}\\
	\scnaddlevel{1}
		\scnciteannotation{Chandrasekaran1999}
	\scnaddlevel{-1};
	\scncite{Chandrasekaran1998}\\
	\scnaddlevel{1}
		\scnciteannotation{Chandrasekaran1998}
	\scnaddlevel{-1};
	\scncite{Fensel1998Reuse}\\
	\scnaddlevel{1}
		\scnciteannotation{Fensel1998Reuse}
	\scnaddlevel{-1};
	\scncite{Kemke2001}\\
	\scnaddlevel{1}
		\scnciteannotation{Kemke2001}
	\scnaddlevel{-1};
	\scncite{Tu1995}\\
	\scnaddlevel{1}
		\scnciteannotation{Tu1995}
	\scnaddlevel{-1};
	\scncite{Trypuz2007}\\
	\scnaddlevel{1}
		\scnciteannotation{Trypuz2007}
	\scnaddlevel{-1};
	\scncite{Fang2019}\\
	\scnaddlevel{1}
		\scnciteannotation{Fang2019}
	\scnaddlevel{-1};
	\scncite{Fensel1997};
	\scncite{McBride2021}\\
	\scnaddlevel{1}
		\scnciteannotation{McBride2021}
	\scnaddlevel{-1};
	\scncite{Crowther2020}\\
	\scnaddlevel{1}
		\scnciteannotation{Crowther2020}
	\scnaddlevel{-1};
	\scncite{McCann1998}\\
	\scnaddlevel{1}
		\scnciteannotation{McCann1998}
	\scnaddlevel{-1};
	\scncite{Yan2014}\\
	\scnaddlevel{1}
		\scnciteannotation{Yan2014}
	\scnaddlevel{-1};
	\scncite{Ansari2018};
	\scncite{Crubezy2004}\\
	\scnaddlevel{1}
		\scnciteannotation{Crubezy2004}
	\scnaddlevel{-1};
	\scncite{Coelho1996}
}
\end{SCn}