\begin{SCn}
    \scnsectionheader{\currentname}
    \begin{scnstruct}
        \scniselement{раздел базы знаний}
        \begin{scnrelfromlist}{дочерний раздел}
            \scnitem{\nameref{intro_sc_code}}
            \scnitem{\nameref{intro_idtf}}
            \scnitem{\nameref{intro_scg}}
            \scnitem{\nameref{intro_scs}}
            \scnitem{\nameref{intro_scn}}
        \end{scnrelfromlist}
        \scntext{аннотация}{Поскольку предлагаемая Вашему вниманию \textit{Публикация Стандарта OSTIS-2021} представляет собой внешний текст основной части \textit{базы знаний} ostis-системы (\textit{Метасистемы IMS.ostis}), необходимо сразу во вводном разделе этого Стандарта пояснить основные принципы, лежащие в основе внутреннего представления \textit{баз знаний} в памяти \textit{ostis-систем}, а также некоторые правила и условные обозначения, используемые в оформлении внешних текстов \textit{баз знаний ostis-систем}.\\Подчеркнем, что все эти принципы, правила и условные обозначения детально рассмотрены в соответствующих разделах \textit{Стандарта OSTIS}, но некоторые из них необходимо пояснить до начала ознакомления с основными положениями \textit{Технологии OSTIS}. Фактически речь идет о кратком руководстве конечных пользователей \textit{ostis-систем}.Прежде, чем рассматривать \textit{внутренний язык*} и \textit{внешние языки*}, используемые \textit{ostis-системами}, необходимо уточнить понятие \textit{информационной конструкции}, понятие \textit{знака}, понятие \textit{текста}, понятие \textit{языка}.\\}
        \scnheader{Предметная область информационных конструкций и языков}
        \scniselement{предметная область}
        \begin{scnhaselementrole}{класс объектов исследования}
            {информационная конструкция}
        \end{scnhaselementrole}
        \begin{scnhaselementrolelist}{класс объектов исследования}
        {
            \scnitem{дискретная информационная конструкция} 
            \scnitem{элемент дискретной информационной конструкции} 
            \scnitem{соответствие, заданное на множестве дискретных информационных конструкций} 
            \scnitem{знак}
            \scnitem{знаковая конструкция}
            \scnitem{язык}
            \scnitem{семантически выделяемый класс дискретных информационных конструкций}
            \scnitem{язык ostis-системы}
            \scnitem{ограничитель}
            \scnitem{идентификатор}
            \scnitem{имя}
            \scnitem{предложение}
            \scnitem{sc.g-текст}
            \scnitem{sc.s-текст}
            \scnitem{sc.n-текст}
        }
        \end{scnhaselementrolelist}
        \begin{scnhaselementrolelist}{исследуемый класс классов первичных объектов исследования}
        {
            \scnitem{отношение, заданное на множестве элементов дискретных информационных конструкций\scnsupergroupsign}
            \scnitem{параметр, заданный на множестве дискретных информационных конструкций\scnsupergroupsign}
            \scnitem{отношение, заданное на множестве знаков\scnsupergroupsign}
            \scnitem{отношение, заданное на множестве знаковых конструкций\scnsupergroupsign}
            \scnitem{класс знаковых конструкций\scnsupergroupsign}
            \scnitem{параметр, заданный на множестве знаковых конструкций\scnsupergroupsign}
            \scnitem{отношение, заданное на множестве языков\scnsupergroupsign}
            \scnitem{параметр, заданный на множестве языков\scnsupergroupsign}
        }
        \end{scnhaselementrolelist}
        \begin{scnhaselementrolelist}{исследуемое отношение}
            \scnitem{операционная семантика информационной конструкции*}
            \scnitem{алфавит*}
            \scnitem{ограничители*}
            \scnitem{разделители*}
            \scnitem{внешний идентификатор*}
            \scnitem{предложения*}
        \end{scnhaselementrolelist}

        \scnheader{ostis-система}
        \scnidtf{компьютерная система, построенная по Технологии OSTIS}
        \scnheader{информационная конструкция}
        \scnidtf{информация}
        \scnidtf{конструкция (структура), содержащая некоторые сведения о некоторых сущностях}
        \scntext{note}{Форма представления ("изображения"{}, материализации), форма структуризации (синтаксическая структура), а также \textit{смысл*} (денотационная семантика) \textit{информационных конструкций} могут быть самыми различными.}\scnheader{дискретная информационная конструкция}
        \scnsubset{информационная конструкция}
        \scntext{explanation}{Каждая \textit{дискретная информационная конструкция}  это \textit{информационная конструкция}, смысл которой задается (1) множеством элементов (синтаксически атомарных фрагментов) этой информационной конструкции, (2) алфавитом этих элементов  семейством классов синтаксически эквивалентных элементов информационной конструкции, (3) принадлежностью каждого элемента информационной конструкции соответствующему классу синтаксически эквивалентных элементов информационной конструкции, (4) конфигурацией связей инцидентности между элементами информационной конструкции.}\scntext{следствие}{Форма представления элементов дискретной информационной конструкции для анализа её смысла не требует уточнения. Главным является (1) наличие простой процедуры выделения (сегментации) элементов дискретной информационной конструкции (2) наличие простой процедуры установления синтаксической эквивалентности разных элементов дискретной информационной конструкции и (3) наличие простой процедуры установления принадлежности каждого элемента дискретной информационной конструкции соответствующему классу синтаксически эквивалентных элементов (т.е. соответствующему элементу алфавита).}
        \scnheader{элемент дискретной информационной конструкции}
        \scnidtf{синтаксически атомарный фрагмент (символ), входящий в состав дискретной информационной конструкции}
        \scntext{note}{Поскольку дискретные информационные конструкции могут иметь общие элементы (атомарные фрагменты) и даже некоторые из них могут быть частями других информационных конструкций, элемент дискретной информационной конструкции может входить в состав сразу нескольких информационных конструкций.}\scnrelto{второй домен}{элемент дискретной информационной конструкции*}
        \scnheader{отношение, заданное на множестве элементов дискретных информационных конструкций\scnsupergroupsign}
        \scnhaselement{элемент дискретной информационной конструкции*}
        \scnidtf{быть элементарным (синтаксически атомарным) фрагментом заданной дискретной информационной конструкции*}
        \scnidtf{быть элементом (атомарным фрагментом) заданной дискретной информационной конструкции*}
        \scnidtf{Бинарное ориентированное отношение, каждая пара которого связывает (1) знак некоторой дискретной информационной конструкции и (2) знак одного из элементов этой дискретной информационной конструкции*}
        \scnrelfrom{второй домен}{элемент дискретной информационной конструкции}
        \scnhaselement{синтаксическая эквивалентность элементов дискретных информационных конструкций*}
        \scnidtf{быть синтаксически эквивалентными элементами (атомарными фрагментами) одной и той же или разных дискретных информационных конструкций, т.е. элементами, принадлежащими одному и тому же классу синтаксически эквивалентных элементов дискретных информационных конструкций*}
        \scnhaselement{инцидентность элементов дискретных информационных конструкций*}
        \scniselement{бинарное ориентированное отношение\scnsupergroupsign}
        \bigskip\scntext{note}{Для \textit{линейных информационных конструкций} -- это последовательность элементов (символов), входящих в состав этих конструкций.\\Для дискретных информационных конструкций, конфигурация которых имеет нелинейный характер, отношение инцидентности их элементов может быть разбито на несколько частных отношений инцидентности, каждое из которых является \uline{подмножеством} объединенного отношения инцидентности. Например, для двухмерных дискретных информационных конструкций это (1) инцидентность элементов информационных конструкций "по горизонтали"{} и (2) инцидентность элементов информационных конструкций по вертикали.}\scnhaselement{неэлементарный фрагмент дискретной информационной конструкции*}
        \scnidtf{быть дискретной информационной конструкцией, которая является \uline{подструктурой} для заданной дискретной информационной конструкции, в состав которой входит (1) подмножество элементов заданной информационной конструкции и, соответственно, (2) подмножество пар инцидентности элементов заданной информационной конструкции.}
        \scnhaselement{алфавит дискретной информационной конструкции*}
        \scnidtf{быть семейством попарно непересекающихся \uline{классов} синтаксически эквивалентных элементов заданной дискретной информационной конструкции*}
        \scnhaselement{первичная синтаксическая структура дискретной информационной конструкции*}
        \scnidtf{быть \textit{графовой структурой}, которая полностью описывает конфигурацию заданной \textit{дискретной информационной конструкции} и которая включает в себя: (1) знаки всех тех классов синтаксически эквивалентных элементов, которым принадлежат элементы описываемой дискретной информационной конструкции, (2) знаки всех элементов (атомарных фрагментов) описываемой информационной конструкции, (3) пары, описывающие инцидентность элементов описываемой информационной конструкции, (4) пары, описывающие принадлежность элементов описываемой информационной конструкции соответствующим классам синтаксически эквивалентных элементов этой информационной конструкции.}
        \scnhaselement{синтаксическая эквивалентность дискретных информационных конструкций*}
        \scntext{definition}{Дискретные информационные конструкции $\bm{Ti}$ и $\bm{Tj}$ являются синтаксически эквивалентными в том и только в том случае, если между конструкцией $\bm{Ti}$ и конструкцией $\bm{Tj}$ существует \uline{изоморфизм}, в рамках которого каждому элементу конструкции $\bm{Ti}$ соответствует синтаксически эквивалентный элемент конструкции $\bm{Tj}$, т.е. элемент, принадлежащий тому же классу синтаксически эквивалентных элементов дискретных информационных конструкций. И наоборот.}\scnhaselement{копия дискретной информационной конструкции*}
        \scnsubset{синтаксическая эквивалентность дискретных информационных конструкций*}
        \scnidtftext{definition}{быть дискретной информационной конструкцией, которая не только синтаксически эквивалентна заданной, но и содержит информацию о форме представления элементов копируемой информационной конструкции*}
        \scnhaselement{семантическая эквивалентность дискретных информационных конструкций*}
        \scntext{definition}{Информационная конструкция $\bm{Ti}$ и информационная конструкция $\bm{Tj}$ являются \uline{семантически эквивалентными} в том и только в том случае, если \uline{каждая} сущность (в том числе, и каждая связь между сущностями), описываемая в информационной конструкции $\bm{Ti}$ описывается также и в информационной конструкции $\bm{Tj}$. И наоборот.}\scnhaselement{семантическое расширение дискретной информационной конструкции*}
        \scntext{definition}{Информационная конструкция $\bm{Tj}$ является семантическим расширением информационной конструкции $\bm{Ti}$ в том и только в том случае, если \uline{каждая} сущность, описываемая в $\bm{Ti}$, описывается также и в $\bm{Tj}$, но обратное неверно.}\scnhaselement{синтаксис информационной конструкции*}
        \scnidtf{синтаксическая структура информационной конструкции*}
        \scnidtf{быть синтаксической структурой для заданной информационной конструкции*}
        \scnidtf{Бинарное ориентированное отношение, каждая пара которого связывает знак некоторой информационной конструкции с синтаксической структурой этой конструкции*}
        \scnidtf{описание того, из каких частей состоит заданная информационная конструкция и как эти части (фрагменты) связаны между собой*}
        \scnrelfrom{первый домен}{информационная конструкция}
        \bigskip\scnhaselement{смысл*}
        \scnidtf{смысл информационной конструкции*}
        \scnidtf{денотационная семантика информационной конструкции*}
        \scnidtf{семантическая (смысловая) структура информационной конструкции*}
        \scnidtf{быть семантической (смысловой) структурой для заданной информационной конструкции*}
        \scnidtf{быть смыслом заданной информационной конструкции*}
        \scnidtf{быть явным (формальным) представлением того, какие сущности описывает данная информационная конструкция и как эти сущности связаны между собой*}
        \scnidtf{Бинарное ориентированное отношение, каждая пара которого связывает некоторую информационную конструкцию с ее смыслом (смысловой структурой)*}
        \scnhaselement{операционная семантика информационной конструкции*}
        \scnidtf{правила трансформации заданной информационной конструкции*}
        \scnidtf{описание того, на основании каких правил можно осуществлять действия по преобразования (обработке, трансформации) заданной информационной конструкции, оставляя ее в рамках класса синтаксически и семантически правильных информационных конструкций*}
        \scnidtf{Бинарное ориентированное отношение, каждая пара которого связывает знак некоторой информационной конструкции со множеством правил ее трансформации*}
        \scnheader{операционная семантика информационной конструкции*}
        \scnrelfrom{второй домен}{операционная семантика информационной конструкции}
        \scnheader{соответствие, заданное на множестве дискретных информационных конструкций}
        \scnhaselement{соответствие между синтаксической структурой информационной конструкции и смыслом этой конструкции*}
        \scnidtf{Множество ориентированных пар, первым компонентом которых является ориентированная пара, состоящая из (1) знака синтаксической структуры некоторой информационной конструкции и (2) знака смысловой структуры этой конструкции, а вторым компонентом которых является множество ориентированных пар, связывающих фрагменты синтаксической структуры заданной информационной конструкции (которые описывают либо структуру фрагментов заданной конструкции, либо связи между фрагментами этой конструкции) с теми фрагментами смысловой структуры заданной информационной конструкции, которые семантически эквивалентны либо синтаксически представленным фрагментам заданной информационной конструкции, либо синтаксически представленным связям между такими фрагментами*}
        \scnsubset{соответствие*}
        \scnheader{параметр, заданный на множестве дискретных информационных конструкций\scnsupergroupsign}
        \scnhaselement{размерность дискретных информационных конструкций\scnsupergroupsign}
        \scnidtf{типология дискретных информационных конструкций, определяемая их размерностью}
        \scntext{explanation}{Это параметр дискретных информационных конструкций, определяющий характер \uline{инцидентности} элементов таких конструкций.}\scnhaselement{линейная информационная конструкция}
        \scnidtftext{explanation}{дискретная информационная конструкция, каждый элемент которой может иметь не более двух инцидентных ему элементов (один слева, другой справа)}
        \scnhaselement{двухмерная информационная конструкция}
        \scnidtftext{explanation}{дискретная информационная конструкция, каждый элемент которой может иметь не более четырех инцидентных ему элементов (слева-справа, сверху-снизу)}
        \scnhaselement{трехмерная информационная конструкция}
        \scnidtftext{explanation}{дискретная информационная конструкция, каждый элемент которой может иметь не более шести инцидентных ему элементов (слева-справа, сверху-снизу, сзади-спереди)}
        \scnhaselement{четырехмерная информационная конструкция}
        \scnidtftext{explanation}{дискретная информационная конструкция, каждый элемент которой может иметь не более восьми инцидентных ему элементов (например, слева-справа, сверху-снизу, сзади-спереди, раньше-позже)}
        \scnhaselement{графовая информационная конструкция}
        \scnidtftext{explanation}{дискретная информационная конструкция, множество элементов которой разбивается на два подмножества  связки и узлы. При этом узлы могут иметь \uline{неограниченное} количество инцидентных им связок}
        \scntext{note}{А в некоторых графовых информационных конструкциях и связки могут иметь неограниченное количество инцидентных им других связок}\scnhaselement{типология дискретных информационных конструкций, определяемая их носителем\scnsupergroupsign}
        \scnhaselement{некомпьютерная форма представления дискретных информационных конструкций}
        \scnsuperset{аудио-сообщение}
        \scnidtf{речевое сообщение}
        \scnidtf{информационная конструкция, представленная в звуковой форме}
        \scnsuperset{информационная конструкция, представленная на языке жестов}
        \scnsuperset{информационная конструкция, представленная в письменной форме}
        \scntext{note}{Конкретный вид носителя для письменной формы представления информации может быть разным  бумага, папирус, береста, камень...}\scnhaselement{файл}
        \scnidtf{компьютерная (электронная) форма (формат) представления и хранения информационных конструкций}
        \scntext{note}{Представление информационных конструкций в виде файлов ориентировано на представление \uline{дискретных} (!) информационных конструкций. Поэтому "файловое"{} представление недискретных информационных конструкций (например, различного рода сигналов) предполагает "дискретизацию"{} таких конструкций, т.е. преобразование их в дискретные. Так преобразуются аудио-сигналы (в частности, речевые сообщения), изображения, видео-сигналы и др.}\scnhaselement{уровень унификации представления синтаксически эквивалентных элементов дискретных информационных конструкций\scnsupergroupsign}
        \scnidtf{уровень "членораздельности"{} дискретных информационных конструкций}
        \scntext{note}{Чем выше уровень унификации представления элементов дискретных информационных конструкций, тем проще реализуется (1) процедура выделения (сегментации) элементов дискретной информационной конструкции, (2) процедура установления синтаксической эквивалентности этих элементов и (3) процедура их распознавания, т.е. процедура установления их принадлежности соответствующим классам синтаксически эквивалентных элементов.}\scnhaselement{дискретная информационная конструкция с низким уровнем унификации представления элементов}
        \scnsuperset{аудио-сообщение}
        \scnsuperset{информационная конструкция, представленная на языке жестов}
        \scnsuperset{рукопись или её копия}
        \scnhaselement{дискретная информационная конструкция с высоким уровнем унификации представления элементов}
        \scnsuperset{печатный текст}
        \scnsuperset{файл}
        \scnheader{знак}
        \scntext{explanation}{фрагмент информационной конструкции, который условно представляет (изображает) некоторую описываемую сущность, которую называют денотатом знака}
        \begin{scnsubdividing}
            \scnitem{знак, являющийся элементом дискретной информационной конструкции}
            \scnitem{знак, являющийся неатомарным фрагментом дискретной информационной конструкции}
        \end{scnsubdividing}
        \scntext{note}{Отсутствие знака, обозначающего некоторую сущность, не означает отсутствие самой этой сущности. Это означает только то, что мы даже не догадываемся о её существовании и, следовательно, не приступили к её исследованию.}\newpage\scnheader{отношение, заданное на множестве знаков\scnsupergroupsign}
        \scntext{note}{Поскольку все знаки являются дискретными информационными конструкциями, множество знаков является областью задания всех отношений, заданных на множестве дискретных информационных конструкций. Тем не менее есть как минимум одно отношение, специфичное для множества знаков.}\scnhaselement{синонимия знаков*}
        \scntext{definition}{Знаки являются синонимичными в том и только в том случае, если они обозначают одну и ту же сущность.}\scntext{note}{Синонимичные знаки могут быть синтаксически эквивалентными, а могут и не быть таковыми.}\scnheader{знаковая конструкция}
        \scnsubset{дискретная информационная конструкция}
        \scnidtftext{explanation}{дискретная информационная конструкция, которая в общем случае представляет собой конфигурацию знаков и специальных фрагментов информационной конструкции, обеспечивающих структуризацию конфигурации знаков  различного вида разделителей и ограничителей}
        \scntext{note}{Для некоторых знаковых конструкций и даже для некоторых языков необходимость в разделителях и ограничителях может отсутствовать.}\scnheader{отношение, заданное на множестве знаковых конструкций\scnsupergroupsign}
        \scnhaselement{знак*}
        \scnidtf{быть знаком для заданной знаковой конструкции*}
        \scnhaselement{разделитель знаковой конструкции*}
        \scnhaselement{разделители знаковой конструкции*}
        \scnidtf{Множество всех разделителей, входящих в состав заданной знаковой конструкции*}
        \scnhaselement{ограничитель знаковой конструкции*}
        \scnhaselement{ограничители знаковой конструкции*}
        \scnhaselement{семантическая смежность знаковых конструкций*}
        \scntext{definition}{Знаковые конструкции $\bm{Ti}$ и $\bm{Tj}$ являются смежными в том и только в том случае, если существуют синонимичные знаки $\bm{Ti}$ и $\bm{Tj}$, один из которых входит в состав конструкции $\bm{Ti}$, а второй  в состав конструкции $\bm{Tj}$}\scnhaselement{конкатенация знаковых конструкций*}
        \scnidtf{декомпозиция заданной знаковой конструкции на последовательность знаковых конструкций*}
        \scnheader{класс знаковых конструкций\scnsupergroupsign}
        \scnhaselement{семантически элементарная знаковая конструкция}
        \scnidtf{знаковая конструкция, описывающая некоторую (одну) связь между некоторыми сущностями}
        \scnhaselement{семантически связная знаковая конструкция}
        \scnidtftext{definition}{знаковая конструкция, которую можно представить в виде конкатенации семантически элементарных знаковых конструкций, каждая из которых семантически смежна предшествующей и последующей семантически элементарной знаковой конструкции}
        \scnheader{параметр, заданный на множестве знаковых конструкций\scnsupergroupsign}
        \scnhaselement{семантическая связность знаковых конструкций\scnsupergroupsign}
        \scnhaselement{семантически связная знаковая конструкция}
        \scnhaselement{семантически несвязная знаковая конструкция}
        \scnhaselementrole{пример}{\scnfilelong{В огороде бузина, а в Киеве дядька}
        }
        \scnhaselement{наличие разделителей и ограничителей\scnsupergroupsign}
        \scnhaselement{знаковая конструкция, содержащая разделители и-или ограничители}
        \scnhaselement{знаковая конструкция без разделителей и ограничителей}
        \newpage\scnheader{язык}
        \scnidtftext{explanation}{класс знаковых конструкций, для которого существуют (1) общие правила их построения и (2) общие правила их соотнесения с теми сущностями и конфигурациями сущностей, которые описываются (отражаются) указанными знаковыми конструкциями}
        \scnidtf{класс знаковых конструкций, эквивалентных с точки зрения правил их построения и правил их семантической интерпретации}
        \begin{scnsubdividing}
            \scnitem{язык, у которого все знаки, входящие в состав его знаковых конструкций, являются элементарными фрагментами этих конструкций\\\scntext{note}{Для языков такого типа существенно упрощаются методы обработки их текстов.}}
            \scnitem{язык, у которого знаки, входящие в состав его знаковых конструкций, в общем случае не являются элементарными фрагментами этих конструкций}
        \end{scnsubdividing}
        \begin{scnsubdividing}
            \scnitem{язык, знаковые конструкции которого содержат разделители и ограничители}
            \scnitem{язык, знаковые конструкции которого не содержат разделителей и ограничителей\\\scntext{следствие}{Знаковые конструкции такого языка состоят \uline{только} из знаков.}}
        \end{scnsubdividing}
        \scnheader{отношение, заданное на множестве языков\scnsupergroupsign}
        \scnidtf{отношение, область определения которого включает в себя множество всевозможных языков}
        \scnhaselement{текст заданного языка*}
        \scnidtf{знаковая конструкция, принадлежащая заданному языку*}
        \scnidtf{синтаксически правильная (правильно построенная) знаковая конструкция заданного языка*}
        \scnidtf{синтаксически корректная и целостная знаковая конструкция для заданного языка*}
        \scneq{{\normalfont(}синтаксически корректная знаковая конструкция для заданного языка* $\cap$ синтаксически целостная знаковая конструкция для заданного языка*{\normalfont)}}
        \scnhaselement{синтаксически корректная знаковая конструкция для заданного языка*}
        \scnidtf{знаковая конструкция, не содержащая синтаксических ошибок для заданного языка}
        \scnhaselement{синтаксически целостная знаковая конструкция для заданного языка*}
        \scnhaselement{синтаксически неправильная знаковая конструкция для заданного языка*}
        \scneq{{\normalfont(}синтаксически некорректная знаковая конструкция для заданного языка* $\cup$ синтаксически нецелостная знаковая конструкция для заданного языка*{\normalfont)}}
        \scnsuperset{синтаксически некорректная знаковая конструкция для заданного языка*}
        \scnsuperset{синтаксически нецелостная знаковая конструкция для заданного языка*}
        \scnhaselement{знание, представленное на заданном языке*}
        \scnidtf{семантически правильный текст заданного языка*}
        \scneq{(семантически корректный текст заданного языка* $\cap$ семантически целостный текст заданного языка*)}
        \scnidtf{истинный текст заданного языка*}
        \scnidtf{истинное высказывание, представленное на заданном языке*}
        \scnhaselement{семантически корректный текст заданного языка*}
        \scnidtf{текст заданного языка, не содержащий семантических ошибок, противоречащих признанным закономерностям и фактам*}
        \scnhaselement{семантически целостный текст заданного языка*}
        \scnidtf{текст заданного языка, содержащий достаточную информацию для установления его истинности*}
        \scnhaselement{семантически неправильный текст для заданного языка*}
        \scneq{(семантически некорректный текст для заданного языка* $\cup$ семантически нецелостный текст для заданного языка*)}
        \scnsuperset{семантически некорректный текст для заданного языка*}
        \scnsuperset{семантически нецелостный текст для заданного языка*}
        \scnhaselement{алфавит*}
        \scnidtf{алфавит заданной информационной конструкции или заданного языка*}
        \scnidtf{семейство классов, синтаксически эквивалентных элементов (элементарных фрагментов) заданной информационной конструкции или информационных конструкций заданного языка*}
        \scnhaselement{семейство классов синтаксически эквивалентных разделителей*}
        \scnidtf{семейство классов синтаксически эквивалентных разделителей, входящих в состав заданной информационной конструкции или в состав информационных конструкций заданного языка*}
        \scnhaselement{семейство классов синтаксически эквивалентных ограничителей*}
        \scnidtf{семейство классов синтаксически эквивалентных ограничителей, входящих в состав заданной информационной конструкции или в состав информационных конструкций заданного языка*}
        \scnhaselement{синтаксис языка*}
        \scnidtf{быть теорией правильно построенных информационных конструкций, принадлежащих заданному языку*}
        \scnidtf{определение понятия правильно построенной информационной конструкции заданного языка*}
        \scnidtf{синтаксические правила заданного языка*}
        \scnidtf{быть синтаксическими правилами заданного языка *}
        \scnidtf{Бинарное ориентированное отношение, каждая пара которого связывает знак некоторого языка с описанием синтаксически выделяемых классов фрагментов конструкций заданного языка, с описанием отношений, заданных на этих классах и с конъюнкцией кванторных высказываний, являющихся синтаксическими правилами заданного языка, то есть правилами, которым должны удовлетворять все синтаксические правильные (правильно построенные) конструкции (тексты) указанного языка*}
        \scntext{note}{При представлении синтаксиса (синтаксических правил) заданного языка используются все те понятия, которые вводятся для представления синтаксических структур информационных конструкций, принадлежащих указанному языку. Это и синтаксически выделяемые классы фрагментов указанных информационных конструкций, и отношения, заданные на множестве таких фрагментов.}\scnrelfrom{второй домен}{синтаксис языка}
        \scnhaselement{описание синтаксических понятий языка*}
        \scnidtf{описание синтаксически выделяемых классов фрагментов конструкций заданного языка*}
        \scnrelfrom{второй домен}{описание синтаксических понятий языка}
        \scnrelto{обобщенное включение}{синтаксис языка}
        \scnhaselement{синтаксические правила языка*}
        \scnidtf{синтаксические правила заданного языка*}
        \scnrelfrom{второй домен}{синтаксические правила языка}
        \scnhaselement{денотационная семантика языка*}
        \scnidtf{быть теорией морфизмов, связывающих правильно построенные информационные конструкции заданного языка с описываемыми конфигурациями описываемых сущностей*}
        \scnhaselement{денотационная семантика языка*}
        \scnidtf{семантические правила заданного языка*}
        \scnidtf{быть семантическими правилами заданного языка *}
        \scnidtf{Бинарное ориентированное отношение, каждая пара которого связывает знак некоторого языка с описанием основных семантических понятий заданного языка и конъюнкцией кванторных высказываний, являющихся семантическими правилами заданного языка, то есть правилами, которым должны удовлетворять семантически правильные \uline{смысловые} информационные конструкции, соответствующие (семантические эквивалентные) синтаксически правильным конструкциям (текстам) заданного языка*}
        \scntext{note}{При формулировке семантических правил заданного языка используются понятия, введенные в рамках базовых онтологий (онтологий самого высокого уровня, в которых рассматриваются самые общие классы описываемых сущностей, самые общие отношения и параметры).}\scnrelfrom{второй домен}{денотационная семантика языка}
        \scnhaselement{описание семантических понятий языка*}
        \scnrelfrom{второй домен}{описание семантических понятий языка}
        \scnhaselement{семантические правила языка*}
        \scnrelfrom{второй домен}{семантические правила языка}
        \bigskip\scnhaselement{семантическая эквивалентность языков*}
        \scnidtf{быть семантически эквивалентными языками*}
        \scntext{definition}{Язык $\bm{Li}$ и язык $\bm{Lj}$ будем считать \textit{семантически эквивалентными языками*} в том и только в том случае, если для каждого текста, принадлежащего языку $\bm{Li}$, существует \textit{семантически эквивалентный текст*}, принадлежащий языку $\bm{Lj}$, и наоборот.}\scnhaselement{семантическое расширение языка*}
        \scnrelboth{обратное отношение}{семантическое сужение языка*}
        \scntext{definition}{Язык $\bm{Lj}$ будем считать \textit{семантическим расширением*} языка $\bm{Li}$ в том и только в том случае, есть ли для каждого текста, принадлежащего языку $\bm{Li}$, существует \textit{семантически эквивалентный текст*}, принадлежащий языку $\bm{Lj}$, но обратное неверно.}\scnhaselement{синтаксическое расширение языка*}
        \scnidtf{быть семантически эквивалентным надмножеством заданного языка*}
        \scntext{definition}{Язык $\bm{Lj}$ будем считать \textit{синтаксическим расширением*} языка $\bm{Li}$ в том и только в том случае, если:\begin{scnitemize}
        \item{$\bm{L_j} \supset \bm{Li}$ (то есть все тексты языка $\bm{Li}$ являются также и текстами языка $\bm{Lj}$, но обратное неверно);}\item{Язык $\bm{Lj}$ и язык $\bm{Li}$ являются \textit{семантически эквивалентными языками*}.}\end{scnitemize}
        }\scnhaselement{синтаксическое ядро языка*}
        \scnidtf{быть синтаксическим ядром заданного языка*}
        \scnidtf{быть семантически эквивалентным подмножеством заданного языка, имеющим минимальную синтаксическую сложность*}
        \scnhaselement{направление синтаксического расширения ядра заданного языка*}
        \scnidtf{быть правилом трансформации информационных конструкций, принадлежащих заданному языку, которое описывает одно из направлений перехода от множества конструкций ядра этого языка ко множеству всех принадлежащих ему информационных конструкций*}
        \scnhaselement{операционная семантика языка*}
        \scnidtf{Бинарное ориентированное отношение, каждая пара которого связывает знак некоторого языка со множеством правил трансформации текстов этого языка*}
        \scnrelfrom{второй домен}{операционная семантика языка}
        \scnhaselement{внутренний язык*}
        \scnidtf{быть внутренним языком для заданной системы, основанной на обработке информации, или заданного множества таких систем*}
        \scnidtf{быть языком внутреннего представления информации в памяти заданной системы, основанной на обработке информации или заданного класса таких систем*}
        \scnhaselement{внешний язык*}
        \scnidtf{быть внешним языком для заданной системы, основанной на обработке информации, или заданного множества таких систем*}
        \scnidtf{быть языком, используемым для обмена информацией заданной системы, основанной на обработке информации, или заданного множества таких систем с другими системами, основанными на обработке информации, (в том числе, и с себе подобными)*}
        \scnhaselement{используемый язык*}
        \scneq{{\normalfont(}внутренний язык* $\cup$ внешний язык*{\normalfont)}}
        \scnidtf{язык, используемый заданной системой, основанной на обработке информации или заданного множества таких систем*}
        \scnidtf{язык, носителем которого является (которым владеет) данная система, основанная на обработке информации}
        \scnhaselement{используемые языки*}
        \scnheader{параметр, заданный на множестве языков\scnsupergroupsign}
        \scnidtf{признак классификации языков}
        \scnidtf{свойство, которым обладают языки}
        \scnidtf{свойство языков, на основании которого можно осуществлять их классификацию}
        \scnidtf{семейство классов эквивалентности языков, трактуемой в контексте того или иного свойства (характеристики), присущего языкам}
        \scnhaselement{семантическая мощность языка\scnsupergroupsign}
        \scnidtf{класс языков, семантически эквивалентных друг другу}
        \scnhaselement{универсальный язык}
        \scnidtf{класс всевозможных универсальных языков}
        \scntext{note}{Очевидно, что все универсальные языки (если они действительно таковыми являются, а не только претендуют на это) семантически эквивалентны друг другу, т.е. имеют одинаковую семантическую мощность.}\scnhaselement{уровень синтаксической сложности представления знаков в текстах языка\scnsupergroupsign}
        \scnhaselement{язык, в текстах которого все знаки представлены синтаксически элементарными фрагментами}
        \scnhaselement{язык, в текстах которого знаки в общем случае представлены синтаксически неэлементарными фрагментами}
        \scnhaselement{использование разделителей и ограничителей в текстах языка\scnsupergroupsign}
        \scnhaselement{язык, в текстах которого не используются разделители и ограничители}
        \scnhaselement{язык, в текстах которого используются разделители и ограничители}
        \scnhaselement{уровень сложности процедуры установления синонимии знаков в текстах языка\scnsupergroupsign}
        \scnhaselement{язык, в рамках каждого текста которого синонимичные знаки отсутствуют}
        \scntext{explanation}{В текстах такого языка знак каждой описываемой сущности входит \uline{однократно}.}\scnhaselement{язык, в рамках которого синонимичные знаки представлены синтаксически эквивалентными фрагментами текстов}
        \scnhaselement{флективный язык}
        \scnidtf{язык, в рамках которого синонимичные знаки могут быть представлены синтаксически неэквивалентными фрагментами, но фрагментами, являющимися модификациями некоторого "ядра"{} этих фрагментов (при склонении и спряжении этих знаков).}
        \scnhaselement{язык, в рамках которого синонимичные знаки в общем случае могут быть представлены синтаксически неэквивалентными текстовыми фрагментами, структура которых носит непредсказуемый характер}
        \scnhaselement{наличие омонимии в текстах языка\scnsupergroupsign}
        \scnhaselement{язык, в текстах которого присутствует омонимия знаков}
        \scnidtf{язык, в текстах которого присутствуют синтаксически эквивалентные, не синонимичные знаки}
        \scnhaselement{язык, в текстах которого омонимия знаков отсутствует}
        \scnheader{семантически выделяемый класс дискретных информационных конструкций}
        \scnhaselement{синтаксическая структура информационной конструкции}
        \scnrelto{второй домен}{синтаксис информационной конструкции*}
        \scnsuperset{первичная синтаксическая структура информационной конструкции}
        \scnsuperset{вторичная синтаксическая структура информационной конструкции}
        \scnhaselement{синтаксис языка}
        \scnhaselement{описание синтаксических понятий языка}
        \scnhaselement{синтаксические правила языка}
        \scnhaselement{денотационная семантика языка}
        \scnhaselement{описание семантических понятий языка}
        \scnhaselement{семантические правила языка}
        \scnhaselement{операционная семантика языка}
        \scnhaselement{смысл}
        \scnrelto{второй домен}{смысл*}
        \scnidtf{смысловая информационная конструкция}
        \scnidtf{смысловое представление информационной конструкции}
        \scnidtf{явное (формальное) представление описываемых сущностей и связей между ними}
        \scnidtf{смысловое представление информации}
        \scntext{note}{Для явного представления описываемых сущностей и связей между ними требуется существенное упрощение синтаксической структуры информационных конструкций}\scnheader{язык ostis-системы}
        \scnidtf{язык, используемый ostis-системами}
        \scnidtf{язык представления информационных конструкций в ostis-системах}
        \scnidtf{Множество языков, которыми владеют\ ostis-системы}
        \scnsubset{формальный язык}
        \scnsubset{универсальный язык}
        \scnrelto{используемые языки}{ostis-система}
        \scnhaselement{SC-код}
        \scnidtf{Semantic Computer Code}
        \scnrelto{внутренний язык}{ostis-система}
        \scniselement{универсальный язык}
        \scnhaselement{SCg-код}
        \scnidtf{Semantic Code graphical}
        \scnidtf{\textit{внешний язык*} ostis-систем, тексты которого представляют собой графовые структуры общего вида с точно заданной \textit{денотационной семантикой*}}
        \scnrelto{внешний язык}{ostis-система}
        \scniselement{универсальный язык}
        \scnhaselement{SCs-код}
        \scnidtf{Semantic Code string}
        \scnidtf{\textit{внешний язык*} ostis-систем, тексты которого представляют собой строки (цепочки) символов}
        \scnrelto{внешний язык}{ostis-система}
        \scniselement{универсальный язык}
        \scnhaselement{SCn-код}
        \scnidtf{Semantic Code natural}
        \scnidtf{\textit{внешний язык*} ostis-систем, тексты которого представляют собой двухмерные матрицы символов, являющиеся результатом форматирования, двухмерной структуризации текстов SCs-кода.}
        \scnrelto{внешний язык}{ostis-система}
        \scniselement{универсальный язык}
        \scntext{explanation}{Для представления \textit{баз знаний ostis-систем} используется целый ряд как \textit{универсальных языков}, так и \textit{специализированных языков}, как \textit{формальных языков}, так и \textit{естественных языков}, как \textit{внутренних языков}, обеспечивающих представление информации в памяти \textit{ostis-систем}, так и \textit{внешних языков}, обеспечивающих представление информации, вводимой в память \textit{ostis-систем}, либо выводимой из этой памяти. \textit{Естественные языки} используются исключительно для представления \textit{файлов}, хранимых в памяти \textit{ostis-системы} и формально специфицируемых в рамках \textit{базы знаний} этой \textit{ostis-системы}. Для эксплуатации \textit{интеллектуальных компьютерных систем}, построенных на основе \textit{SC-кода}, кроме способа абстрактного внутреннего представления баз знаний (\textit{SC-кода}) потребуются несколько способов внешнего изображения абстрактных \textit{sc-текстов}, удобных для пользователей и используемых при оформлении исходных текстов \textit{баз знаний} указанных интеллектуальных компьютерных систем и исходных текстов фрагментов этих \textit{баз знаний}, а также используемых для отображения пользователям различных фрагментов \textit{баз знаний} по пользовательским запросам. В качестве таких способов внешнего отображения \textit{sc-текстов} и предлагаются указанные выше внешние языки ostis-систем (\textit{SCg-код}, \textit{SCs-код} и  \textit{SCn-код}).Для описания перечисленных \textit{языков}, используемых \textit{ostis-системами}, в каждом из них мы выделим \textit{ядро языка*}, которое является \textit{семантически эквивалентным языком*} для соответствующего языка и имеет минимальную синтаксическую сложность. Описание каждого из указанных языков строится как описание нескольких направлений синтаксического расширения выделенного \textit{языка-ядра}.}\scntext{note}{Все основные внешние формальные языки, используемые ostis-системами (\textit{SCg-код}, \textit{SCs-код}, \textit{SCn-код}) являются различными вариантами внешнего представления текстов внутреннего языка ostis-систем -- SC-кода. Указанные языки являются универсальными и, следовательно, \textit{семантически эквивалентными языками*}.}\newpage\scntext{note}{Каждая ostis-система может приобрести способность использовать любой внешний язык (как универсальный, так и специализированный, как естественный, так и искусственный), если синтаксис и денотационная семантика этого языка будут описаны в памяти ostis-системы на ее внутреннем языке (SC-коде).}\scnheader{следует отличать*}
        \begin{scnhaselementset}
            \scnitem{семантическое расширение языка*\\\scnidtf{расширение семантической мощности языка*}}
            \scnitem{синтаксическое расширение языка*\\\scnidtf{расширение синтаксиса языка при сохранении его семантической мощности*}}
        \end{scnhaselementset}
        \bigskip
        \begin{scnhaselementset}
            \scnitem{синтаксическое расширение языка*}
            \scnitem{направление синтаксического расширения ядра заданного языка*\\\scnidtf{одно из (или группа) правил синтаксической трансформации текстов заданного языка, расширяющих множество синтаксически правильных текстов этого языка}
            \scntext{note}{Таких направлений синтаксического расширения заданного языка может быть несколько.}}
        \end{scnhaselementset}
        \bigskip
        \begin{scnhaselementset}

            \item{
                \begin{scnset}
                    \scnitem{синтаксис дискретной информационной конструкции*}
                    \scnitem{синтаксическая структура дискретной информационной конструкции}
                \end{scnset}~\\
                \scniselement{следует отличать*}
            }
            \item{
                \begin{scnset}
                \scnitem{первичная синтаксическая структура дискретной информационной конструкции*}
                \scnitem{первичная синтаксическая структура дискретной информационной конструкции}
                \end{scnset}~\\
                \scniselement{следует отличать*}
            }
            \item{
                \begin{scnset}
                    \scnitem{смысл*}
                    \scnitem{смысл\\\scnidtf{смысловая информационная конструкция}}
                \end{scnset}~\\
                \scniselement{следует отличать*}
            }
            \item{
                \begin{scnset}
                    \scnitem{синтаксис языка*}
                    \scnitem{синтаксис языка}
                \end{scnset}~\\
                \scniselement{следует отличать*}
            }
            \item{
                \begin{scnset}
                    \scnitem{денотационная семантика языка*}
                    \scnitem{денотационная семантика языка}
                \end{scnset}~\\
                \scniselement{следует отличать*}
            }
            \item{
                \begin{scnset}
                    \scnitem{операционная семантика языка*}
                    \scnitem{операционная семантика языка} 
                \end{scnset}~\\
                \scniselement{следует отличать*}
            }
        \end{scnhaselementset}
        \bigskip
        \begin{scnhaselementset}
            \scnitem{смысловое представление знака*\\
            	\scnidtf{Бинарное ориентированное отношение, каждая пара которого связывает фрагмент синтаксической структуры некоторой дискретной информационной конструкции (точнее, фрагмент смыслового представления этой синтаксической структуры) с элементом смыслового представления исходной дискретной информационной конструкции}}
            \scnitem{смысл*\\
            	\scnidtf{смысловое представление дискретной информационной конструкции*}}
            \scnitem{денотационная семантика языка}
        \end{scnhaselementset}
        \bigskip
        \begin{scnhaselementset}
            \scnitem{
                информационная трансформация\\\scnidtf{трансформация информационной конструкции}
                \scnidtf{действие по преобразованию (обработке, трансформации) информационной конструкции}
                \scnsubset{действие}
            }
            \scnitem{класс однотипных информационных конструкций}
            \scnitem{правило выполнения однотипных информационных трансформаций\\\scnidtf{обобщенная спецификация класса однотипных информационных трансформаций, описывающая обобщенное условие выполнения произвольной трансформации указанного класса, ее результат и соответствие между условием и реультатом}}
            \scnitem{программа выполнения однотипных информационных трансформаций\\\scnidtf{обобщенная декомпозиция произвольной информационной трансформации заданного класса на систему взаимосвязанных трансформаций более низкого уровня}}
            \scnitem{агент выполнения однотипных информационных трансформаций}
        \end{scnhaselementset}
        \bigskip\scnheader{следует отличать*}
        \begin{scnhaselementset}
            \scnitem{денотационная семантика*}
            \scnitem{описание денотационной семантики*}
            \scnitem{операционная семантика*}
            \scnitem{описание операционной семантики*}
        \end{scnhaselementset}
        \scnheader{следует отличать*}
        \begin{scnhaselementset}
            \scnitem{денотационная семантика знака}
            \scnitem{денотационная семантика текста}
            \scnitem{денотационная семантика языка}
        \end{scnhaselementset}
        \scnheader{следует отличать*}
        \begin{scnhaselementset}
            \scnitem{синтаксис ~\\ 
            	\scnidtf{синтаксис языка}~\\
            	\scnidtf{отношение, связывающее тексты некоторого языка и соответствующие тексты того же или другого языка, описывающие синтаксическую структуру этих текстов}}
            \scnitem{синтаксис*\\
            	\scnidtf{отношение, связывающее язык и его синтаксис}}
            \scnitem{синтаксическая конструкция текста\\
            	\scnidtf{информационная конструкция, описывающая синтаксическую структуру текста некоторого языка}}
            \scnitem{описание синтаксиса\\
            	\scnidtf{информационная конструкция, описывающая синтаксис некоторого языка и его свойства, включают правила построения синтаксически корректных конструкций данного языка}}
        \end{scnhaselementset}
        \scnheader{язык ostis-системы}
        \scntext{note}{Следует отличать:
        \begin{scnitemize}
            \item{саму описываемую сущность;}
            \item{текст, являющийся описанием некоторой сущности;}
            \item{текст, являющийся описанием некоторого другого текста, а возможно и самого себя (т.е. текст может быть описываемой сущностью);}
            \item{знак (обозначение) описываемой сущности в рамках заданного текста;}
            \item{обозначение описываемой сущности в sc-тексте (это всегда sc-элемент того или иного вида);}
            \item{коммуникативный (внешний для ostis-системы) уникальный (основной) идентификатор (чаще всего строковый идентификатор-имя), обозначающий соответствующую описываемую сущность и являющийся внешним идентификатором (именем) для соответствующего синонимичного ему sc-элемента. Такие идентификаторы взаимно однозначно соответствуют sc-элементам, которые имеют такие идентификаторы;}
            \item{вспомогательные (неосновные) внешние идентификаторы sc-элементов. Такие идентификаторы могут обладать и свойством омонимии (когда один идентификатор соответствует нескольким sc-элементам) и синонимии (когда разные идентификаторы соответствует одному sc-элементу);}
            \item{обозначение описываемой сущности в sc.g-тексте -- это всегда графически представленный sc.g-элемент, являющийся \uline{изображением} соответствующего sc-элемента;}
            \item{обозначение описываемой сущности в sc.s-предложении и в sc.n-предложении -- это всегда строка символов (либо \uline{омонимичное} изображение sc-коннекторов различного семантического типа, либо \uline{основной} строковый идентификатор, соответствующий некоторому sc-элементу, либо выражение, являющееся \uline{неатомарным} идентификатором, содержащим некоторую информацию о соответствующей именуемой сущности).}
        \end{scnitemize}
        Подчеркнем, что каждое \uline{обозначение} описываемой сущности в SCg-коде, SCs-коде, SCn-коде рассматривается нами как \uline{изображение} соответствующего ему (синонимичного ему) sc-элемента, обозначающего ту же описываемую сущность. Таким образом, указанные языки (SCg-код, SCs-код; SCn-код) рассматриваются нами как различные варианты изображения текстов SCg-кода.}\scntext{note}{Для формального описания различного рода языков, включая рассматриваемые нами языки (SCg-код, SCs-код, SCn-код) используется целый ряд метаязыковых понятий.Перечислим некоторые из них: \textit{идентификатор}, \textit{класс синтаксически эквивалентных идентификаторов}, \textit{имя}, \textit{простое имя}, \textit{выражение}, \textit{внешний идентификатор*}, \textit{алфавит*}, \textit{разделители*}, \textit{ограничители*}, \textit{предложения*}}\scntext{note}{Синтаксис \textit{языков представления знаний в ostis-системах} может быть формально описан различными способами. Так, например, можно использовать метаязык Бэкуса-Наура для описания синтаксиса SCs-кода или его расширение для описания синтаксиса SCn-кода. Однако значительно более логично и целесообразно описывать синтаксис всех форм внешнего отображения sc-текстов с помощью самого SC-кода. Такой подход позволит ostis-системам самостоятельно понимать, анализировать и генерировать тексты указанных языков на основе принципов, общих для любых форм внешнего представления информации, в том числе нелинейных.}\scnheader{алфавит*}
        \scnidtf{быть алфавитом для заданного множества текстов*}
        \scnidtf{быть семейством максимальных множеств синтаксически однотипных элементарных (атомарных) фрагментов текстов, принадлежащих заданному множеству текстов*}
        \scnheader{ограничители*}
        \scnidtf{Отношение, связывающее заданный класс информационных конструкций с соответствующим классом их ограничителей}
        \scnidtf{быть ограничителями, используемыми в заданном множестве информационных конструкций*}
        \scnheader{ограничитель}
        \scnsuperset{sc.g-ограничитель}
        \scnsuperset{sc.s-ограничитель}
        \scnsuperset{sc.n-ограничитель}
        \scnsuperset{ограничитель, используемый в ея-файлах ostis-систем}
        \scnheader{SCg-код}
        \scnrelfrom{ограничители}{sc.g-ограничитель}
        \scnidtf{Множество ограничителей, используемых в sc.g-текстах}
        \scnheader{разделители*}
        \scnidtf{быть разделителями, используемыми в заданном множестве информационных конструкций*}
        \scnrelfrom{второй домен}{разделитель}
        \scnsuperset{sc.g-разделитель}
        \scneq{sc.g-коннектор}
        \scnsuperset{sc.s-разделитель}
        \scnsuperset{sc.n-разделитель}
        \scnsuperset{разделитель, используемый в ея-файлах ostis-систем}
        \scnheader{идентификатор}
        \scnsuperset{sc.s-идентификатор}
        \scnidtf{cтруктурированный знак соответствующей (обозначаемой) сущности, который чаще всего представляет собой строку (цепочку символов), которую будем называть именем соответствующей сущности.}
        \scntext{note}{В формальных текстах (в том числе текстах SC-кода, SCg-кода, SCs-кода, SCn-кода) основные используемые идентификаторы не должны быть омонимичными, то есть должны \uline{однозначно} соответствовать идентифицируемым сущностям. Следовательно, каждая пара идентификаторов, имеющих \uline{одинаковую} структуру, должны обозначать одну и ту же сущность.}\scnheader{следует отличать*}
        \begin{scnhaselementset}
            \scnitem{идентификатор\\\scnidtf{Множество всевозможных конкретных \uline{экземпляров}, конкретных вхождений идентификаторов, имеющих различную структуру, во всевозможные тексты}}
            \scnitem{класс синтаксически эквивалентных идентификаторов\\\scnidtf{класс идентификаторов, имеющих одинаковую структуру}
            \scnidtf{Семейство всевозможных множеств, каждое из которых является максимальным множеством синтаксически эквивалентных идентификаторов}}
        \end{scnhaselementset}
        \scnheader{имя}
        \scnsubset{идентификатор}
        \scnidtf{строковый идентификатор}
        \scnidtf{идентификатор, представляющий собой строку (цепочку) символов}
        \begin{scnsubdividing}
            \scnitem{простое имя\\\scnidtf{атомарное имя}
            \scnidtf{имя, в состав которого другие имена не входят}}
            \scnitem{выражение\\\scnidtf{неатомарное имя}}
        \end{scnsubdividing}
        \scnsuperset{sc.s-идентификатор}
        \scnheader{внешний идентификатор*}
        \scniselement{отношение}
        \scnidtf{Бинарное ориентированное отношение, каждая связка (sc-дуга) которого связывает некоторый элемент с файлом, содержимым которого является внешний идентификатор (чаще всего, имя), соответствующий указанному элементу}
        \scnidtf{быть внешним идентификатором*}
        \scnidtf{внешний идентификатор sc-элемента*}
        \scnrelfrom{второй домен}{идентификатор}
        \scntext{note}{Понятие внешнего идентификатора является понятием относительным и важным для ostis-систем, поскольку внутреннее для ostis-систем представление информации (в виде текстов SC-кода) оперирует не идентификаторами описываемых сущностей, а знаками, структура которых никакого значения не имеет}\scnheader{следует отличать*}
        \begin{scnhaselementset}
            \scnitem{sc-элемент, обозначающий файл ostis-системы}
            \scnitem{sc.g-элемент, обозначающий файл ostis-системы}
            \scnitem{простой sc.s-идентификатор, обозначающий файл ostis-системы\\\scnidtf{простое имя файла ostis-системы}}
            \scnitem{изображение файла ostis-системы, ограниченное sc.g-рамкой}
            \scnitem{изображение файла ostis-системы, ограниченное sc.n-рамкой}
            \scnitem{изображение строки символов, ограниченное квадратными скобками}
        \end{scnhaselementset}
        \scnheader{следует отличать*}
        \begin{scnhaselementset}
            \scnitem{файл-экземпляр}
            \scnitem{файл-класс\\\scnidtf{файл, обозначающий класс файлов-экземпляров, синтаксически эквивалентных заданному образцу}}
        \end{scnhaselementset}
        \scnheader{предложения*}
        \scnidtf{быть множеством всех предложений заданного текста, не являющихся встроенными предложениями, то есть предложениями, входящими в состав других предложений*}
        \scnrelfrom{второй домен}{предложение}
        \scnheader{предложение}
        \scntext{explanation}{минимальный семантически целостный фрагмент текста, представляющий собой конфигурацию знаков, входящих в этот фрагмент и связываемых между собой отношениями инцидентности (в частности, отношением непосредственной последовательности в строке), а также различного вида разделителями и ограничителями}\scnheader{sc.g-текст}
        \scnidtf{текст SCg-кода}
        \begin{scnsubdividing}
            \scnitem{семантически связный sc.g-текст\\\scnidtf{sc.g-текст, который семантически эквивалентен \uline{связному} sc-тексту}}
            \scnitem{семантически несвязный sc.g-текст}
        \end{scnsubdividing}
        \begin{scnsubdividing}
            \scnitem{синтаксически связный sc.g-текст}
            \scnitem{синтаксически несвязный sc.g-текст}
        \end{scnsubdividing}
        \scnheader{sc.s-текст}
        \scnidtf{текст SCs-кода}
        \begin{scnsubdividing}
            \scnitem{семантически связный sc.s-текст\\\scnidtf{sc.s-текст, который семантически эквивалентен \uline{связному} sc-тексту}}
            \scnitem{семантически несвязный sc.s-текст}
        \end{scnsubdividing}
        \begin{scnsubdividing}
            \scnitem{синтаксически связный sc.s-текст}
            \scnitem{синтаксически несвязный sc.s-текст}
        \end{scnsubdividing}
        \scnheader{sc.n-текст}
        \scnidtf{текст SCn-кода}
        \begin{scnsubdividing}
            \scnitem{семантически связный sc.n-текст\\\scnidtf{sc.n-текст, который семантически эквивалентен \uline{связному} sc-тексту}}
            \scnitem{семантически несвязный sc.n-текст\\}
        \end{scnsubdividing}
        \begin{scnsubdividing}
            \scnitem{синтаксически связный sc.n-текст\\}
            \scnitem{синтаксически несвязный sc.n-текст\\}
        \end{scnsubdividing}
        \scnheader{следует отличать*}
        \begin{scnhaselementset}
            \scnitem{денотационная семантика*}
            \scnitem{описание денотационной семантики*}
            \scnitem{операционная семантика*}
            \scnitem{описание операционной семантики*}
        \end{scnhaselementset}
        \bigskip
        \begin{scnhaselementset}
            \scnitem{денотационная семантика знака}
            \scnitem{денотационная семантика текста}
            \scnitem{денотационная семантика языка}
        \end{scnhaselementset}
        \bigskip
        \begin{scnhaselementset}
            \scnitem{синтаксис~\\
            	\scnidtf{синтаксис языка}~\\
            	\scnidtf{отношение, связывающее тексты некоторого языка и соответствующие тексты того же или другого языка, описывающие синтаксическую структуру этих текстов}}
            \scnitem{синтаксис*~\\
            	\scnidtf{Отношение, связывающее язык и его синтаксис}}
            \scnitem{синтаксическая структура текста\\
            	\scnidtf{информационная конструкция, описывающая синтаксическую структуру текста некоторого языка}}
            \scnitem{описание синтаксиса~\\
            	\scnidtf{информационная конструкция, описывающая синтаксис некоторого языка и его свойства, включая правила построения синтаксически корректных конструкций данного языка}}
        \end{scnhaselementset}
        \bigskip
    \end{scnstruct}
    \scnendcurrentsectioncomment
\end{SCn}